%FOR PDFLATEX USE ONLY
\documentclass[a4paper,12pt]{article}

\usepackage{amssymb,amsmath} %math symbols

\usepackage[margin=1cm]{geometry} %paper geometry

\usepackage[utf8]{inputenc} %allows unicode (including russian) source file
\usepackage[russian]{babel} %docment in russian-style
\usepackage[utf8]{inputenc}
\usepackage[unicode]{hyperref} %links inside of the text
\usepackage[pdftex]{graphicx} %includegraphics pictures
\usepackage{cmlgc} %bold text

\usepackage{array} %arrays

\usepackage{wrapfig}
\usepackage{array}
\usepackage{lipsum}
\usepackage{esvect}
\usepackage{hyperref}

\usepackage{subfig}
\usepackage{calc}
\usepackage{pgfplots,tikz,circuitikz}
\usepackage{tkz-euclide}

\newenvironment{changemargin}[2]{%
\begin{list}{}{%
\setlength{\topsep}{0pt}%
\setlength{\leftmargin}{#1}%
\setlength{\rightmargin}{#2}%
\setlength{\listparindent}{\parindent}%
\setlength{\itemindent}{\parindent}%
\setlength{\parsep}{\parskip}%
}%
\item[]}{\end{list}}

\begin{document}
\section{1.7}

\section{2.3}
\begin{center}
	Элементарные частицы и их свойства \\
\end{center}
\begin{scriptsize}
\begin{center}
\begin{tabular}{cccccccc}
\hline
& Электрический & Цветной & Барионное& Спин & Магнитный  & Изоспин & Внутренняя\\
& заряд &заряд& число & & момент  && четность\\
Протон & +1 &"белый"& 1 &  1/2 & 1,41060679736(60)$\cdot10^{-26}$  & 1/2& 1\\
Нейтрон & 0 &"белый"& 1 &  1/2 & -9,6623651(23)$\cdot10^{-27}$ & -1/2& 1\\
\hline
& Электрический &Цветной& Барионное& Спин & Слабый & Изоспин& Четность\\
& заряд &заряд& число & & гиперзаряд && \\
Пион &$\pm1$ & "белый"&0&0&0, -2, -1&$\pm1$& -
\\
\hline
& Электрический &Цветной& Лептонное & Спин & Магнитный && Внутренняя   \\
& заряд &заряд& число & & момент && четность\\
Электрон & -1 &0 & 1 & 1/2 & -9,274009994(57)$\cdot10^{-24}$ && 1
\\
\hline
& Электрический &Цветной& Лептонное & Спин &&& \\
& заряд & заряд & число &&&&  \\
Мюон & -1 &0& 1 & 1/2 &&&
\\
\hline
& Электрический &Цветной& Лептонное & Спин & Кол-во спиновых   \\
& заряд &заряд& число & & состояний \\
$\tau$-лептон & -1 &0& 1 & 1/2 & 2
\\
\hline
& Электрический &Цветной& Лептонное & Спин & Слабый  \\
& заряд &заряд& число & & гиперзаряд \\
Нейтрино & 0 & 0 & 1 & 1/2 & -1
\\
\hline
& Электрический &Цветной& Барионное & Спин  \\
& заряд &заряд& число & \\
Кварк & кратен e/3 & r, g, b & 1/3 & 1/2 
\\
\hline
& Электрический & Цветной && Спин & Количество спиновых  \\
& заряд & заряд &&& состояний\\
W$^+$- базон & +1& 0&& 1&3\\
W$^-$- базон& -1 &0 &&1&3\\
Z -  базон&0 &0 && 1&3\\
\hline
& Электрический &Цветной&& Спин &Кол-во спиновых&& Внутренняя\\
& заряд &заряд&& & Состояний&& четность \\
Глюон& 0& $r\bar r, g\bar g, b\bar b, r\bar g, r\bar b, g\bar b$,&& 1& 2&& -
\\
\hline
&Эликтрический& Цветной&& Спин& Кол-во спиновых& Спиральность& Внутренняя\\
&заряд& заряд&&& состояний&&четность\\
Фотон &0& 0&& 1& 2& $\pm$1& -
\\
\hline
&Электрический& Цветной&& Спин&&& Четность\\
&заряд& заряд\\
Базон Хиггса &0& 0&& 0&&& +1\\
\hline
\end{tabular}
\end{center}
\end{scriptsize}

Электрический заряд - квантовое число, определяющее способность частиц принимать участие в электромагнитном взаимодействие.\\
Цветной заряд — квантовое число, приписываемое глюонам и кваркам, которые участвуют в сильном взаимодействии. Цветов три: «красный», «зелёный» и «синий», хотя эти названия не имеют никакого отношения к цветам, которые мы видим в повседневной жизни, также существуют три антицвета.\\
Спин — собственный момент импульса элементарных частиц. Спин измеряется в единицах $\hbar$ (постоянной Дирака) и равен $\hbar$J, где J — характерное для каждого сорта частиц целое или полуцелое положительное число — так называемое спиновое квантовое число, которое и приведено в таблице.\\
Изоспин — квантовое число, определяющая число зарядовых состояний адронов.\\
Магнитный момент — квантовое число, характеризующее магнитные свойства частицы. Измеряется в Дж/Тл.\\
Барионное число — квантовое число, определяющее количество барионов (элементарных частиц, состоящих из трёх кварков) в системе.\\
Лептонное число — разность числа лептонов (частиц с полуцелым спином, не участвующих в сильном взаимодействии) и антилептонов в данной системе.\\
Спиральность — квантовое число, используемое при описании элементарных частиц, движущихся со скоростью света или близкой к ней.

\end{document}