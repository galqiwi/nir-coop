\documentclass[10pt,a4paper]{article}
\usepackage[utf8]{inputenc}
\usepackage[russian]{babel}
\usepackage[OT1]{fontenc}
\usepackage{amsmath}
\usepackage{amsfonts}
\usepackage{amssymb}
\usepackage{graphicx}
\DeclareGraphicsExtensions{.pdf,.png,.jpg}
\usepackage[left=2cm,right=2cm,top=2cm,bottom=2cm]{geometry}

\begin{document}

Задача Кеплера, это когда:\\
$\mathbf {F} ={\frac {k}{r^{2}}}\mathbf {\hat {r}}$\\
$V(r)={\frac {k}{r}}$, где $V(r)$ - потенциальная энергия\\
Нетрудно заметить, что в модели атома Бора для водорода эти условия выполняются.\\
Таким образом, можно применить к движению электрона в атоме водорода и получить:\\
1. электрон движется по эллипсу(окружности)(1 закон Кеплера)\\
2. за равное время заметает равные секторальные площади(2 закон Кеплера)\\
3. 3 закон Кеплера\\
А исходя из этих законов можно посчитать период обращения электрона на разных энергетических уровнях, различные параметры траектории и т.д.\\

\end{document}