\documentclass[a4paper,12pt]{article} % тип документа
% report, book

% Рисунки
\usepackage{graphicx}
\usepackage{wrapfig}

\usepackage{hyperref}
\usepackage[rgb]{xcolor}
\hypersetup{				% Гиперссылки
    colorlinks=true,       	% false: ссылки в рамках
	urlcolor=blue           % на URL
}

%  Русский язык

\usepackage[T2A]{fontenc}			% кодировка
\usepackage[utf8]{inputenc}			% кодировка исходного текста
\usepackage[english,russian]{babel}	% локализация и переносы

% Математика
\usepackage{amsmath,amsfonts,amssymb,amsthm,mathtools} 
\usepackage{wasysym}

\begin{document} % начало документа

5. 9
\\ Принцип <<космической цензуры>> был сформулирован в 1970 году Роджером Пенроузом в следующей образной форме: <<Природа питает отвращение к голой сингулярности>>. Формулировка Пенроуза предполагает, что пространство-время в целом является глобально гиперболическим. Горизонт событий чёрной дыры является световой поверхностью, образующие которой при продолжении их в будущее никогда не пересекаются между собой. В таких областях становится неприменимым базовое приближение большинства физических теорий, в которых пространство-время рассматривается как гладкое многообразие без края. Часто в гравитационной сингулярности величины, описывающие гравитационное поле, становятся бесконечными или неопределёнными. К таким величинам относятся, например, скалярная кривизна или плотность энергии в сопутствующей системе отсчёта. В классической чёрной дыре в сингулярности сила гравитации настолько велика, что свет не может покинуть горизонт событий и, таким образом, объекты внутри горизонта событий, включая саму чёрную дыру, не могут наблюдаться непосредственно.
\\ До осени 2017 года были основания сомневаться в его абсолютной правильности (например, коллапс пылевого облака с большим угловым моментом приводит к <<голой сингулярности>>, но неизвестно, стабильно ли это решение уравнений Эйнштейна относительно малых возмущений начальных данных). В своей работе, опубликованной в октябре 2017 года, математики Михалис Дафермос и Джонатан Лак доказали, что сильная форма космической цензуры, относящаяся к странной структуре чёрных дыр, неверна.
\\ В 2019 году американский физик Уильям Ист численно смоделировал коллапс вытянутого эллипсоида, заполненного пылью, и показал, что гипотеза космической цензуры в этом случае не нарушается, как это предполагалось ранее. Оказалось, что вместо голых сингулярностей при коллапсе в материи формируются каустики, а все сингулярности в конце концов накрываются горизонтом событий.
\\ Несмотря на то, что Общая теория относительности (ОТО) описывает большинство наблюдаемых гравитационных явлений, она не является полной. Одна из самых больших проблем ОТО — существование гравитационных сингулярностей, в которых кривизна пространства-времени обращается в бесконечность. В частности, такие сингулярности возникают в центре черной дыры Шварцшильда. Очевидно, что для описания процессов в окрестности сингулярности нужно учитывать не только гравитационные, но и квантовые эффекты. Грубо говоря, чем ближе мы подходим к сингулярности, тем больше рождается виртуальных частиц, которые, в свою очередь, искривляют пространство-время и разрушают исходную геометрию. К сожалению, физики до сих пор так и не построили Теорию Всего, которая объединила бы ОТО и Квантовую теорию поля. Поэтому никто не знает, как выглядит сингулярность на самом деле.

\end{document} % конец документа