\documentclass{article}

\usepackage[T2A]{fontenc}
\usepackage[utf8]{inputenc}
\usepackage[russian]{babel}
\usepackage{amsmath,amsfonts,amssymb}
\usepackage[textwidth=17cm,textheight=24cm]{geometry}
\usepackage{cancel}

\DeclareMathOperator{\tr}{tr}

\begin{document} 
В кулоновском поле у частицы сохраняется энергия, момент имупульса и вектор Рунге-Ленца. 

Энегрия: 

\begin{equation*}
E = \frac{M}{2} \dot{r}^2 + \frac{L^2}{2Mr^2} - \frac{k}{r} = const
\end{equation*}
Где $\mathbf {L} = [\mathbf r \times \mathbf P ]$ - момент импульса частицы, $M$ - ее приведенная масса, $U = - \frac{k}{r}$ - потенциальная энергия, а $r$ - радиус вектор из начала координат. 

Разберемся с моментом импульса тела, т.к. момент силы, действующей на частицу нулевой, то:
\begin{equation*}
\mathbf {M} = [\mathbf r \times \mathbf F ] = 0 ; \quad  [\mathbf r \times \frac{d \mathbf P}{dt} ] = 0
\end{equation*}
Следовательно момент импульса $L$:
\begin{equation*}
\frac{d}{dt} [\mathbf r \times \mathbf P ] = [\mathbf r \times \frac{d \mathbf P}{dt} ] +  [\frac{d\mathbf r}{dt} \times \mathbf P] = [\mathbf r \times \frac{d \mathbf P}{dt} ] = 0
\end{equation*}
Т.е. $\frac{d\mathbf L}{dt} = 0$, значит $L = [\mathbf r \times \mathbf P ]= const$

Определим вектор Рунге-Ленца, как:
\begin{equation*}
\mathbf A = [\mathbf P \times \mathbf L] - Mk\widehat{\mathbf r}
\end{equation*}
Данный вектор также сохраняется в кулоновском поле:
\begin{equation*}
\mathbf F =\frac{d \mathbf P}{dt} = f(r)\frac{\mathbf r}{|r|} = f(r)\widehat{\mathbf r}
\end{equation*}
Т.к. $L = const$
\begin{equation*}
\frac{d}{dt} [\mathbf P \times \mathbf L ] = \frac{d \mathbf P}{dt} \times \mathbf L =  f(r)\widehat{\mathbf r} \times \left(\mathbf r \times M \frac{d\mathbf r}{dt} \right) =  f(r)\frac{M}{r}\left[ \mathbf r\left(\mathbf r \frac{d\mathbf r}{dt}\right) - r^2\frac{d\mathbf r}{dt} \right]
\end{equation*}
Откуда получаем:
\begin{equation*}
\frac{d}{dt} [\mathbf P \times \mathbf L ] = -M f(r) r^2 \left[\frac{1}{r}\frac{d\mathbf r}{dt} - \frac{\mathbf r}{r^2}\frac{dr}{dt}  \right]
\end{equation*}
При $f(r) = -\frac{k}{r^2}$, последнее выражение равно:
\begin{equation*}
\frac{d}{dt} [\mathbf P \times \mathbf L ] = Mk\frac{d}{dt}\left( \frac{\mathbf r}{r}\right) = \frac{d}{dt}(Mk\widehat{\mathbf r})
\end{equation*}
Тогда изменение вектора:
\begin{equation*}
\frac{d}{dt}\mathbf A = \frac{d}{dt} [\mathbf P \times \mathbf L ] - \frac{d}{dt}(Mk\widehat{\mathbf r}) = 0
\end{equation*}
Отсюда получаем, что в кулоновском поле существует однородность времени, изотропность пространства, а также некоторая симметрия вектора Рунге-Ленца
\end{document}