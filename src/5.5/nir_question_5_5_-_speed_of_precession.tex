\documentclass[a4paper, 12pt]{article}

\usepackage{cmap}
\usepackage{mathtext}
\usepackage[T2A]{fontenc}
\usepackage[utf8]{inputenc}
\usepackage[english, russian]{babel}
\usepackage{icomma}
\usepackage{amsmath,amsfonts,amssymb,amsthm,mathtools}
\usepackage[textwidth=18cm,textheight=25cm]{geometry}
\usepackage{tikz}
\usepackage{pgfplots}
\pgfplotsset{compat=1.3}

\title{\textbf{Когда и почему постоянна скорость прецессии?}}

\begin{document}

\maketitle

Вынужденная прецессия гироскопа вызывается моментом внешних сил, поворачивающих момент импульса гироскопа, согласно уравнению
\begin{equation}\label{1}
    \dot{\vec{L}} = \vec{M}.
\end{equation}
Момент импульса гироскопа, равный $\vec{L} = \hat{I} \vec{\omega}$, можно разложить по друм осям:
$$
\vec{L} = I_{||} \vec{\omega}_{||} + I_{\perp} \vec{\omega}_{\perp},
$$
где $I_{||} \vec{\omega}_{||}$ -- компонента вектора импульса, направленная вдоль оси фигуры гироскопа, а $I_{\perp} \vec{\omega}_{\perp}$ -- вдоль оси, перпендикулярной к оси фигуры.

В приближённой теории гироскопа мы считаем, что $I_{||} \omega_{||} \gg I_{\perp} \omega_{\perp}$ (гироскоп быстро вращается вокруг оси фигуры), и, как следствие, $\vec{L} \approx I_{||} \vec{\omega}_{||} \approx I_{||} \vec{\omega}$, т.е. $\vec{L}$ и $\vec{\omega}$ сонаправлены.
\\
Пусть внешняя сила $\vec{F}$ приложена к точке, лежащей на оси фигуры гироскопа; $\vec{r}$ -- радиус-вектор, проведённый от точки опоры гироскопа к точке приложения силы $\vec{F}$. Момент этой силы будет равен $\vec{M} = [\vec{r},\vec{F}]$. Тогда, в силу уравнения (\ref{1}), $\dot{\vec{L}}$ будет перпендикулярен к оси фигуры, т.е. под влиянием момента внешней силы будет изменяться направление $\vec{L}$, а не его модуль -- это и есть явление прецессии. Найдём угловую скорость $|\vec{\omega}_{пр}|$ этого вращения.
\\
Так как $\vec{L}$ изменяется не по модулю, а только по направлению, то можем представить, что $\vec{L}$ -- радиус-вектор некоторой точки. Тогда линейная скорость этой точки $\dot{\vec{L}}$ будет равна $[\vec{\omega}_{пр}, \vec{L}]$, отсюда получаем
$$
[\vec{\omega}_{пр}, \vec{L}] = \vec{M}.
$$
Пусть $\vec{\omega}_{пр}$ составляет угол $\alpha$ с $\vec{L}$. Тогда
$$
\omega_{пр} L sin \alpha = M = r F sin \alpha
$$
$$
\omega_{пр} = \frac{r F}{L} = \frac{r F}{I_{||} \omega}.
$$
Отсюда ясно, что $\omega_{пр} = const$, когда $\frac{r F}{I_{||} \omega} = const$.

Приведённое выше рассуждение справедливо для быстро вращающегося гироскопа, т.е. $\omega_{||} \gg \omega_{\perp}, \omega_{пр}$. При медленном вращении нутационные колебания (колебания оси около её среднего направления) могут быть довольно заметными и, слагаясь с прецессией, существенно изменить движение оси: конец оси будет описывать ясно видимую волнообразную или петлеобразную кривую, то отклоняясь от вертикали, то приближаясь к ней.

\end{document}
