\documentclass[a4paper, 12pt]{article}
\usepackage[T2A,T1]{fontenc}
\usepackage[utf8]{inputenc}
\usepackage[english, russian]{babel}
\usepackage{graphicx}
\usepackage[hcentering, bindingoffset = 10mm, right = 15 mm, left = 15 mm, top=20mm, bottom = 20 mm]{geometry}
\usepackage{multirow}
\usepackage{lipsum}
\usepackage{amsmath, amstext}
\usepackage{siunitx}
\usepackage{mathrsfs}
\usepackage{subcaption}
\usepackage{wrapfig}
\usepackage{adjustbox}
\usepackage{enumerate, indentfirst, float}
\usepackage{capt-of, svg}
\usepackage{icomma}
\newenvironment{bottompar}{\par\vspace*{\fill}}{\clearpage}

\begin{document}

    \begin{center}
    \textsc{\Large Вопрос 4.5 НИР}\\[0.5cm] 
    \end{center}

    

При движении в полях $ U = \frac{\alpha}{r}$ (ньютоновские поля тяготения и кулоновские электростатические поля) с любым знаком $\alpha$  имеется интеграл движения, специфический именно для этого поля. Легко проверить, непосредственным вычислением, что величина
$$[\textbf{vM}] + \frac {\alpha \textbf{r}}{r} = const (1).$$
Действительно, её полная производная по времени равна
$$[\mathbf{\dot{v}M}] + \frac {\alpha \textbf{v}}{r} - \frac {\alpha \textbf{r} (\textbf{vr})}{r^3},$$
или, подставив $\textbf{M} = m[\textbf{rv}]:$
$$m\mathbf{r(v\dot{v})} - m\mathbf{v(r\dot{v})} + \frac {\alpha \textbf{v}}{r} - \frac {\alpha \textbf{r} (\textbf{vr})}{r^3};$$
положив здесь согласно уравнениям движения $ m\mathbf{\dot{v}} = \frac {\alpha \textbf{r}}{r^3}$, мы найдём, что это выражение обращается в нуль.
Сохраняющийся вектор (1) направлен вдоль большой оси от фокуса к перигелию, а по величине равен $\alpha e$, где $e$ - эксцентриситет орбиты.

    

\end{document}
