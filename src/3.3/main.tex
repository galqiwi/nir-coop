%FOR PDFLATEX USE ONLY
\documentclass[a4paper,14pt]{article}

\usepackage{amssymb,amsmath} %math symbols

\usepackage[margin=2cm, bottom=2cm]{geometry} %paper geometry

\usepackage[utf8]{inputenc} %allows unicode (including russian) source file
\usepackage[russian]{babel} %docment in russian-style
\usepackage[utf8]{inputenc}
\usepackage[unicode]{hyperref} %links inside of the text
\usepackage[pdftex]{graphicx} %includegraphics pictures
\usepackage{cmlgc} %bold text

\usepackage{array} %arrays

\usepackage{cancel}
\usepackage{wrapfig}
\usepackage{array}
\usepackage{lipsum}
\usepackage{esvect}
\usepackage{longtable}
\usepackage{verbatim} 
\usepackage{multirow}
\usepackage{hyperref}
\usepackage{mathtools}
\usepackage{subfig}
\usepackage{calc}
\usepackage{pgfplots,tikz,circuitikz}
\usepackage{tkz-euclide}
\usepackage{gensymb}



\begin{document}

\section*{3.3}
\begin{center}
	\LARGE{\textbf{Уравнения движения и лагранжиан электромагнитного поля.}}
\end{center}

Свойства поля характеризуются 4-потенциалом $A_i$ в кажоый точке пространства-времени. Он определен с точностью до $\frac{\delta f}{\delta x^i}$, где f -- произвольная функйия от координат и времени. Привычные нам величины -- векторы электрического и магнитных полей выражаются через тензор электромагнитного поля:
\begin{equation*}
	F_{\mu \nu} = \begin{pmatrix}
		0 & E_x & E_y & E_z \\
		-E_x & 0 & -B_z & B_y \\
		-E_y & B_z & 0 & -B_x \\
		-E_z & -B_y & B_x & 0
	\end{pmatrix},
\end{equation*}
где тензор электромагнитного поля определяется через 4-потенциал, как:
\begin{equation*}
	\mathrm{F}_{\mu \nu} = \frac{\delta A_\nu}{\delta x^\mu} - \frac{\delta A_\mu}{\delta x^\nu}.
\end{equation*}
Лагранжиан электромагнитного поля равен:
\begin{equation*}
	L_{em} = -\frac{1}{4}F_{\mu \nu}F_{\mu \nu} + \frac{1}{c}j_\mu A_\mu.
\end{equation*}
Уравнения движения электромагнитного поля (которые, кстати, выводятся из Лагранжиана) известны под названием "уравнения Максвелла", и представляют из себя систему из 4 уравнений, описывающих эволюцию электромагнитного поля со временем (в си):
\begin{equation*}
	\begin{cases}
		\nabla \cdot \mathbf{E} = \frac {\rho} {\varepsilon_0},\\
		\nabla \cdot \mathbf{B} = 0,\\
		\nabla \times \mathbf{E} = -\frac{\partial \mathbf{B}} {\partial t},\\
		\nabla \times \mathbf{B} = \mu_0\left(\mathbf{J} + \varepsilon_0 \frac{\partial \mathbf{E}} {\partial t} \right).
	\end{cases}
\end{equation*}


\end{document}