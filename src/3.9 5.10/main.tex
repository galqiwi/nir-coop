\documentclass[12 pt]{article}
\usepackage[utf8]{inputenc}
\usepackage[russian]{babel}
\usepackage{amsmath}
\usepackage{txfonts}
\usepackage{graphicx}  % Для вставки рисунков
\usepackage{wrapfig} % Обтекание рисунков и таблиц текстом
\usepackage{pgfplots}
\usepackage{filecontents}
\usepackage{tikz}
\usepackage{floatrow}
\floatsetup[table]{capposition=top}
\newcommand{\harpoon}{\overset{\rightharpoonup}}
\usepackage[left=2cm,right=2cm,
    top=2cm,bottom=2cm]{geometry}

\title{Исследование вынужденной регулярной прецессии гироскопа}
\author{Архипов Павел}

\begin{document}

\section{3.9}

Оценим время, которое требуется фотону, чтобы вылететь из Солнца, таким образом: положим томсоновское рассеяние на свободном электроне самым распространённым типом рассеяния фотонов в недрах Солнца, его эффективное сечение $\sigma = 7 \cdot 10^{−29} \text{ м}^2$.

Для оценки положим концентрацию электронов $n_e$ не зависящей от глубины. Длина свободного пробега фотона $\lambda = \frac{1}{n_e \sigma} = 2 \text{ см}$.

Вследствие многократного рассеяния, путь фотона выглядит как случайное блуждание. Тогда время, за которое фотон переместится на радиус Солнца $R$, читается так:
$$T = \frac{R^2}{\lambda c} = 10^3 \text{ лет}.$$

Полученная оценка является несколько заниженной. В действительности, конечно, томсоновский механизм реализуется в глубоких слоях Солнца, где концентрация электронов существенно выше средней, а при более низких температурах возможны более эффективные взаимодействия с веществом.

Источник: задача со сборов к международной олимпиаде по астрофизике.

\section{5.10}

\begin{enumerate}
    \item Изотермическая атмосфера.
    Принимая температуру и ускорение свободного падения постоянными, а атмосферу - находящейся в динамическом равновесии, получим:
    $$\rho (h) = \rho_0 \exp\left[-\frac{\mu g h}{RT}\right].$$
    Эта модель не учитывает изменения температуры атмосферы с высотой.
    
    \item Адиабатическая атмосфера.
    В модели адиабатической атмосферы в динамическом равновесии, плотность зависит от высоты так:
    $$\rho (h) = \rho_0 \left(1 - \frac{\gamma - 1}{\gamma} \frac{\mu g h}{R T_0}\right)^{1/(\gamma - 1)}.$$
    В этой модели температура линейно падает с высотой. В этой модели атмосфера должна закончиться на тридцати километрах над уровнем моря. Однако, на гораздо больших высотах наблюдаются метеоры и серебристые облака. Эта модель не учитывает отклонение зависимости температуры от высоты от прямой.
    
    \item Обе формулы выше плохо описывают плотность атмосферы на больших высотах. Температура атмосферы зависит от высоты нетривиальным образом. Для более хорошего описания атмосферы существуют более точные модели, например, International Standard Atmosphere или US Standard Atmosphere. Фактически, эти модели представляют собой кусочно заданные функцию, учитывающие сложное поведение зависимости температуры от высоты.
    
    Источник: Wikipedia.
    
\end{enumerate}

\end{document}
