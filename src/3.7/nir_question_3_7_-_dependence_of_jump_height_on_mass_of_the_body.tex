\documentclass[a4paper, 12pt]{article}

\usepackage{cmap}
\usepackage{mathtext}
\usepackage[T2A]{fontenc}
\usepackage[utf8]{inputenc}
\usepackage[english, russian]{babel}
\usepackage{graphicx}
\usepackage{icomma}
\usepackage{amsmath,amsfonts,amssymb,amsthm,mathtools}
\usepackage[textwidth=18cm,textheight=25cm]{geometry}
\usepackage{tikz}
\usepackage{pgfplots}
\usepackage{longtable}
\pgfplotsset{compat=1.3}

\title{Зависимость высоты прыжка от массы тела (3.7)}
\author{Гаврилов Дмитрий, Б02-926}
\date{2 октября 2019}

\begin{document}

\maketitle

\section{Сопротивление воздуха}

Определим вклад сопротивления воздуха в зависимость высоты прыжка от массы тела. Для этого рассмотрим вертикальный прыжок человека. По 2-му закону Ньютона относительно вертикальной оси $Ox$, направленной вверх: $ma_x(t) = -mg - kv_x^{\alpha}(t)$, где $\alpha = {1; 2}$ в зависимости от вида обтекания. Тогда нам нужно сравнить слагаемые $-mg$ и $-kv^{\alpha}$. Для этого возьмём предельную скорость падения человека, равную $v_{пр} \approx 60 м/с$ (измеренная предельная скорость парашютиста), тогда по 2-му закону Ньютона предельная сила сопротивления $F_{пр}$ равна по модулю силе $mg$. Взяв $v_0 \approx 3 м/с$ (можно оценить по собственному прыжку), найдём отношение максимальной силы сопротивления при данном прыжке $F_0$ к предельной:
\\
в случае турбулентного обтекания $\frac{F_0}{mg} = \frac{F_0}{F_{пр}} = \frac{k v_0^2}{k v_{пр}^2} = (\frac{v_0}{v_{пр}})^2 \approx \frac{1}{400}$;
\\
в случае ламинарного обтекания $\frac{F_0}{mg} = \frac{F_0}{F_{пр}} = \frac{k v_0}{k v_{пр}} = \frac{v_0}{v_{пр}} \approx \frac{1}{20}$.
\\
\smallskip
Можем видеть, что при любом обтекании даже максимальная сила сопротивления мала по сравнению с силой тяжести, и ей можно пренебречь.
\\
Стоит отметить, что для животных, способных прыгать с большой начальной скоростью (например, небольших насекомых) сила сопротивления может играть значительную роль.

\section{Размеры тела}

Примем за $l$ характерный размер тела. Очевидно, его масса пропорциональна кубу размера: $m \sim l^3$. Сила $F$, которую способны создать мышцы тела, пропорциональна давлению $p$, возникающему вследствие веса тела, и площади сечения мышц $S$. Давление имеет верхний предел, поскольку начиная с некоторого значения массы тела кости и мышцы не выдержат нагрузки, поэтому можем считать давление постоянным для любой массы тела. Площадь $S$ же пропорциональна квадрату размера: $S \sim l^2$, отсюда $F \sim l^2$. Тогда работа, совершаемая мышцами при прыжке и пропорциональная $Fl$ (мышцы "проходят" большее расстояние при большем размере), будет так же, как и масса, пропорциональна кубу размера. Отсюда можно сделать вывод, что начальная кинетическая энергия тела на единицу массы, т.е. $v_0^2/2$, не зависит от размера. Как следствие, $v_0$ не зависит от размеров тела, так же как и высота прыжка.
\\
Данное рассуждение справедливо, если для людей верны приведенные пропорциональности. Если человек, например, обладает излишней массой, то начальная скорость прыжка будет все же меньше, нежели у человека такого же роста с нормальной массой тела. Как следствие, такой человек сможет прыгнуть не так высоко.

\end{document}
