\documentclass{article}
\usepackage{amsmath}
\usepackage{mathtext}
\usepackage[T1,T2a]{fontenc}
\usepackage[utf8]{inputenc}
\usepackage[english, bulgarian, russian]{babel}
\usepackage{tikz}
\usepackage{pgfplots}
\usepackage[export]{adjustbox}
\usepackage{siunitx}
\usepackage{booktabs}
\usepackage{pgfplotstable}
\usepackage[left=2cm,right=2cm,
    top=2cm,bottom=2cm,bindingoffset=0cm]{geometry}
    
    
\sisetup{
  round-mode          = places, % Rounds numbers
  round-precision     = 2, % to 2 places
}


\begin{document}
  \pagenumbering{gobble}
  \maketitle
  \newpage
  \pagenumbering{arabic}
  
  

\section{В чём состоит (утверждает) Общая теория относительности?}

	\subsection{Введение}

		Создание специальной теории относительности в значительной мере изменило вид классической механики. Однако, в ней постулат относительности всё ещё ограничивается необходимостью выбора исключительного класса инерциальных систем отсчёта, двигающихся друг относительно друга равномерно и прямолинейно. Таким образом, становится вполне понятно, что общековариантные уравнения должны быть лишены этого ограничения, т. е. в общей теории должна быть возможность произвольного выбора координат.

		Так как в гравитационных полях пробные тела вне зависимости от их массы двигаются одинаковым образом, то вполне логично рассматривать такое движение как простое движение вдоль геодезических линий в криволинейном пространстве. 

		Общая теория относительности позволяет описать данные эффекты, будучи записанной при этом с помощью общековариантных уравнений, что решает вышпоставленные проблемы. 

	\subsection{Движение в гравитационном поле}

		Свойства пространства задаются метрическим тензором $ g_{ij} $ . Если тензор $ g_{ij} $
		получен в результате решения полевых уравнений, то уравнения движения частицы определяются варьированием действия для частицы. 

		\begin{align*}
			\delta S & = -mc \delta \int ds = 0 \\
		\end{align*}

		Из чего получаем уравнения движения:

		\begin{align*}
			D & u^i = 0 \\
			& или      \\
			\frac{d^2 x^i}{ds^2} + & Г_{jk}^i \frac{dx^j}{ds} \frac{dx^k}{ds} = 0 \\
		\end{align*}

		где D - ковариантный диффиренциал, а $ Г_{jk}^i $ - символ Кристоффеля: 

		\begin{align*}
			Г_{jk}^i = \frac{1}{2} g^{im} & ( \partial_{j} g_{km} + \partial_{k} g_{jm} - \partial_{m} g_{jk} ) \\
		\end{align*}

		Решая их, можно получить уравнения движения частицы

	\subsection{Уравнения поля}	
		
		Для получения уравнений поля следует с начала определить тензор кривизны пространства (тензор Римана). Рассмотрим изменение вектора $ A_k $ при параллельном переносе его вдоль бесконечно малого контура:

		\begin{align*}
			\Delta A_k = \oint \delta A_k = \oint Г_{kl}^i A_i dx^l 
			= \frac{1}{2} [ & \partial_l (Г_{km}^i A_i) - \partial_m (Г_{kl}^i A_i)] \Delta S^{lm} = \\
			= \frac{1}{2} [   \partial_l (Г_{km}^i) A_i - \partial_m (Г_{kl}^i) A_i + & (Г_{km}^i) \partial_l A_i - (Г_{kl}^i) \partial_m A_i ] \Delta S^{lm} = \\
			= \frac{1}{2} R^i_{klm} & A_i \Delta S^{lm}
		\end{align*}

		где 

		\begin{align*}
			R^i_{klm} = \partial_l (Г_{km}^i) - \partial_m (Г_{kl}^i)  + Г_{km}^n Г_{nl}^i - Г_{kl}^n Г_{nm}^i
		\end{align*}

		Соответственно тензором Риччи и символом Риччи называются свертки тензора кривизны:

		\begin{align*}
			R_{km} = R^l_{klm} = \partial_l (Г_{km}^l) - \partial_m (Г_{kl}^l) & + Г_{km}^n Г_{nl}^l - Г_{kl}^n Г_{nm}^l \\
			R  = g^{km} R_{km} &
		\end{align*}

		Действие $ S_g $ для гравитационного поля должно быть выражено в виде некоторого скаллярного интеграла по инвариантному гиперобъёму:

		\begin{align*}
			S_g = \int G \sqrt{-g} d^4 x
		\end{align*}

		причём G не должно содержать никаких производных $ g_{ij} $ выше первого порядка, т.е. G - функция только $ g_{ij} $ и $ Г_{ij}^k $ . Требуемый истинный скаляр составить при данных условиях невозможно, однако, следующий инвариантный интеграл представим в виде:

		\begin{align*}
			\int R \sqrt{-g} d^4 x = \int G \sqrt{-g} d^4 x + \int \partial_i ( \sqrt{-g} w^i ) d^4 x
		\end{align*}

		где последний член при варьировании действия зануляется. Опусканием полной производной в выражении $ \sqrt{-g} \: R $ получаем искомый вид G:

		\begin{align*}
			G = g^{ik} ( Г_{il}^m Г_{km}^l - Г_{ik}^l Г_{lm}^m )
		\end{align*}

		Теперь, для получения уравнений поля остаётся проварьировать суммарное действие системы:

		\begin{align*}
			( \delta S_m + \delta S_g ) = 0
		\end{align*}

		Для $ \delta S_g $ элементарными преобразованиями получаем:

		\begin{align*}
			\delta S_g \propto \delta \int R \sqrt{-g} d^4 x =  \int ( R_{ij} - \frac{1}{2} g_{ij} R) \delta g^{ik} \sqrt{-g} d^4 x
		\end{align*}


		Вариация действия для материи выражается через тензор энергии-импульса следующим образом: 

		\begin{align*}
			\delta S_m = \frac{1}{2c} \int T_{ij} \delta g^{ij} \sqrt{-g} d^4 x
		\end{align*}

		Таким образом, искомые уравнения Эйнштейна имеют вид:

		\begin{align*}
			R_{ij} - & \frac{1}{2} g_{ij} R = \chi T_{ij} \\
			& или               \\
			R_{ij} = & \chi ( T_{ij} - \frac{1}{2} g_{ij} T )
		\end{align*}

		где $\chi = \frac{8 \pi k_G }{ c^4 }$ - гравитационная постоянная Эйнштейна, зависящая от выбора системы единиц. \\

		Теперь, имея уравнения поля мы можем качественно рассчитывать поле произвольных физических систем. \\

		Стоит заметить, что уравнения Эйнштейна нелинейны, т.е. в ОТО не справедлив принцип суперпозиции.

	\subsection{Законы сохранения}

		В специальной теории относительности 4-импульс системы можно выразить через тензор энергии-импульса следующим образом:

		\begin{align*}
			P^i = \frac{1}{c} \int T^{ik} dS^k
		\end{align*}

		В ОТО к выражению нужно добавить псевдотензор энергии-импульса гравитационного поля t^{ik}:

		\begin{align*}
			P^i = & \frac{1}{c} \int (-g)(T^{ik} + t^{ik}) dS^k \\
			t^{ik} = \frac{ \chi }{-2g} \{ \: \partial_l \partial_m & [(-g)(g^{ik} g^{lm} - g^{il} g^{km})] + g^{ik} R - 2R^{ik} \}
		\end{align*}

		Путём элементарных вычислений получаем выражение $t^{ik}$ через символы Кристоффеля:

		\begin{align*}
			t^{ik} = \frac{ \chi }{2} \{ ( g^{il} g^{km} - g^{ik} g^{lm} )( 2Г_{lm}^n Г_{np}^p - Г_{lp}^n Г_{mn}^p - Г_{ln}^n Г_{mp}^p ) + \\
			+ g^{il} g^{mn} ( Г_{lp}^k Г_{mn}^p + Г_{mn}^k Г_{lp}^p - Г_{np}^k Г_{lm}^p - Г_{lm}^k Г_{np}^p ) + \\
			+ g^{kl} g^{mn} ( Г_{lp}^i Г_{mn}^p + Г_{mn}^i Г_{lp}^p - Г_{np}^i Г_{lm}^p - Г_{lm}^i Г_{np}^p ) + \\
			+ g^{lm} g^{np} ( Г_{ln}^i Г_{mp}^k - Г_{lm}^i Г_{np}^k ) \}
		\end{align*}

		В силу того, что определённый таким образом $t^{ik}$ симметричен, сохраняется тензор момента импульса системы:

		\begin{align*}
			M^{ij} = \int [ x^i ( T^{jk} + t^{jk} ) - x^j ( T^{ik} + t^{ik} )](-g)dS_k
		\end{align*}

		Что также позволяет опрделить центр инерции.

	\subsection{Точные решения}

		В силу нелинейности уравнений Эйнштейна их решение предствляет собой весьма сложную задачу, и получение каждого точного решения является весьма сложной, с точки зрения математики, задачей. Одним из простейших решений, является метрика Шварцшильда, описывающая поле снаружи невращающегося сферически симметричного незаряженного тела. Приведём её в качестве примера:

		\begin{align*}
			g_{SC} =
			\left( \begin{array}{cccc}
				(1-\frac{r_s}{r}) & 0                       & 0    & 0                \\
				0                 & -(1-\frac{r_s}{r})^{-1} & 0    & 0                \\
				0                 & 0                       & -r^2 & 0                \\
				0                 & 0                       & 0    & -r^2 sin^2 \theta 
			\end{array} \right)
		\end{align*}

		В данной метрике очевидны такие эффекты как замедление хода часов и изменение геометрических размеров предметов в гравитационном поле, не присутствующие в теории тяготения Ньютона. 

\end{document}