\documentclass{article}
\usepackage[utf8]{inputenc}
\usepackage[russian]{babel}
\usepackage[dvips]{graphicx}
\usepackage{natbib}
\usepackage{bm}

\begin{document}

\section*{3.6. Когда у уравнений $\bm{a_3x^3 + a_2x^2 + a_1x + a_0 = 0}$ и $\bm{b_2x^2 + b_1x + b_0 = 0}$ есть общее решение?}
    Многочлены в условии обозначим как $a(x)$ и $b(x)$.
    Общие корни двух многочленов это корни их наибольшего общего делителя, поэтому у данных многочленов будет общий корень тогда и только тогда, когда будет корень у их НОДа. Для многочленов третьей и второй степени с данными коэфициентами НОД может быть быстро найден алгоритмом Эвклида, поиск потребует не более двух делений в столбик. В нашем случае остаток от деления первого многочлена на второй (в предположении $b_2\neq0$) равен 
    $$
    \left(a_1+a_3\frac{b_1^{2}}{b_2^{2}}-a_3\frac{b_0}{b_2}-a_2\frac{b_1}{b_2}\right) x + \left(a_0+a_3\frac{b_0 b_1}{b_2^{2}}-a_2\frac{b_0}{b_2}\right) = c_1 x + c_0 = c(x)
    $$
    Если $c(x) = 0$, то общий корень есть тогда и только тогда, когда у $b(x)$ вообще есть корень. Если $c_1 = 0$, но $c_0 \neq 0$, то общего корня нет. В противном случае корень есть тогда и только тогда, когда 
    $$(b(x), c(x))=0\Leftrightarrow b\left(-\frac{c_0}{c_1}\right)=0\Leftrightarrow b_2\frac{c_0^{2}}{c_1^{2}}-b_1\frac{c_0}{c_1}+b_0=0.$$
    

\section*{5.4 Как находят экзопланеты?}
    Один из способов -- метод транзитов. Проводятся наблюдение за яркостью звёзд, чтобы среди тех, которые изменяют её периодично, выявить звёзды с экзопланетами (если лоскость орбиты \textit{почти} содержит луч зрения, то планета будет периодически проходить по диску звезды). С помощью этого метода обнаружено подавляющее большинство известных на сегодня экзопланет, в частности, именно этим методом в основном искал планеты телескоп Кеплер. Второй способ, с отрывом являвшийся наиболее используемым до появления т. Кеплер -- наблюдение за лучевой скоростью звезды (из-за движения звезды вокруг центра масс системы лучевая скорость меняется со временем, из-за чего линии в её спектре движутся). Точность лучших современных спектрометров в некоторых случаях позволяет обнаружить флуктуации лучевой скорости даже меньшие, чем 3$\frac{m}{s}$. Поиск планет при микролинзировании -- ещё один способ, появившийся относительно недавно. Если одна звезда выступает как грав. линза для света другой далёкой звезды, то возможная планета у линзирующей звезды изменит изображение. Наконец, существует метод прямого наблюдения -- когда планету удаётся обнаружить на снимке (звезда при этом закрывается коронографом). Существуют и другие методы, которые, однако, практически не используются. Методы, естественно, комбинируются: не редко земным спектрометром измеряют лучевую скорость у звезды, попавшей в список кандидатов на наличие планеты благодаря транзитным наблюдениям на космическом телескопе.


\end{document}
