\documentclass{article}
\usepackage{amsmath}
\usepackage{mathtext}
\usepackage[T1,T2a]{fontenc}
\usepackage[utf8]{inputenc}
\usepackage[english, bulgarian, russian]{babel}
\usepackage{tikz}
\usepackage{pgfplots}
\usepackage[export]{adjustbox}
\usepackage{siunitx}
\usepackage{booktabs}
\usepackage{pgfplotstable}
\usepackage{latexsym}
\usepackage[left=2cm,right=2cm,
    top=2cm,bottom=2cm,bindingoffset=0cm]{geometry}
    
    
\sisetup{
  round-mode          = places, % Rounds numbers
  round-precision     = 2, % to 2 places
}


\begin{document}
  \pagenumbering{gobble}
  \maketitle
  \newpage
  \pagenumbering{arabic}
  
\section{В чём состоит (утверждает) теория Калуцы-Клейна? }

\subsection{ Введение }

	К 1916 году Альберт Эйнштейн заканчивает создание Общей теории относительности, получившей впоследствии небывалый успех. Теория смогла предсказать множество не имеющих до этого момента обьяснения явлений, таких как "прецессия перигелия орбиты Меркурия" и "космологическое красное смещение". Элегантно обьединяя в себе уравнения гравитационного поля и уравнения движения, теория, тем не менее, никак не обьясняла наличие других фундаментальных взаимодействий. \\
	В апреле 1919 года А. Эйнштейн получает письмо малоизвестного математика Теодора Калуцы, в котором тот предлагает модель, претендующую на объединение ОТО и классической теории Максвелла. Последующая разработка этой теории множеством знаменитых учёных XX века привела к созданию первой внутренне непротиворечивой теории объединения, однако, не имевшевшей экспериментального подтверждения. Тем не менее, в дальнейшем идеи Калуцы нашли продолжение в дальнейшем.

\subsection{ Основные идеи }

	 Основной идеей является предложение дополнить криволинейное пространство Эйнштейна компактным пятым измерением:

	\begin{align*}
	 	\tilde g_{ab} =
		\left( \begin{array}{cc}
			g_{ \mu \nu } + \phi^2 A_{\mu} A_{\nu} & \phi^2 A_{\mu}    \\
			\phi^2 A_{ \nu }                       & \phi^2            \\
		\end{array} \right)
	\end{align*}

	Где $ \tilde g_{ab} $ - пятимерный метрический тензор с латинскими индексами соответственно, $g_{ \mu \nu }$ - обычный четырёхмерный метрический тензор ОТО, индексируемый греческим алфавитом. Как мы увидим впоследствии, величины $A_{\mu}$ следует отождествлять с компонентами 4-потенциала электромагнитного поля.
	Тогда ковариантный метрический тензор

	\begin{align*}
	 	\tilde g^{ab} =
		\left( \begin{array}{cc}
			g^{ \mu \nu } & - A^{\mu}                                                  \\
			- A^{ \nu }   &  g_{\alpha \beta} A^{\alpha} A^{\beta} + \frac{1}{\phi^2}  \\
		\end{array} \right)
	\end{align*}

	Таким образом, метрика будет иметь вид:

	\begin{align*}
	 	ds^2 = \tilde g_{ab} dx^a dx^b = g_{\mu \nu} dx^{\mu} dx^{\nu} + \phi^2 (A_{\nu} dx^{\nu} + dx^5)^2
	\end{align*}

\subsection{Уравнения поля в вакууме}

	Для упрощения уравнений обычно принимается условие цилиндра:

	\begin{align*}
	 	\frac{ \partial \tilde g_{ab} }{\partial x^5} = 0
	\end{align*}

	По аналогии с ОТО вводим символы Кристоффеля, тензор Риччи и действие для поля:

	\begin{align*}
	 	\tilde{Г}_{bc}^a= \frac{1}{2} \tilde g^{ad} ( \partial_b \tilde g_{dc}& + \partial_c \tilde g_{db} - \partial_d \tilde g_{bc}) \\
	 	\tilde{R}_{ab} = \partial_c \tilde{Г}_{ab}^c - \partial_b \tilde{Г}_{ca}^c & + \tilde{Г}_{cd}^c \tilde{Г}_{ab}^d - \tilde{Г}_{bd}^c \tilde{Г}_{ac}^d \\
	 	S_g = \int \tilde{R}_{ab} & \tilde{g}^{ab} \sqrt{ |\tilde{g}| } d^5 x
	\end{align*}

	Варьрование этой части действия даёт

	\begin{align*}
	 	\tilde{R}_{ab} - \frac{1}{2} \tilde g_{ab} \tilde R = 0
	\end{align*}

	Или в четырёхмерной форме:

	\begin{align*}
	 	{R}_{\mu \nu} - \frac{1}{2} g_{\mu \nu} R = \frac{1}{2} \phi^2 ( g^{\alpha \beta } F_{\mu \alpha} F_{\nu \beta} - \frac{1}{4} g_{\mu \nu} F_{\alpha \beta} F^{\alpha \beta} ) + \frac{1}{\phi}(\nabla_{\mu} \nabla_{\nu} \phi - g_{\mu \nu} \Box \phi)
	\end{align*}

	Где $ F_{\mu \nu} = \partial_{\mu} A_{\nu} - \partial_{\nu} A_{\mu} $

	Заметим, что при выборе $\phi$ в качестве постоянной левая часть уравнения переходит в тензор энергии - импульса электромагнитного поля, т. е. уравнение находится в полном соответствии с уравнениями Эйнштейна в вакууме.

\subsection{Уравнения движения}

	Определим 5-скорость:

	\begin{align*}
	 	\tilde{U}^a = \frac{dx^a}{ds}
	\end{align*}

	Уравнения движения примут вид:

	\begin{align*}
	 	\tilde{U}^b \nabla_b \tilde{U}^a = \frac{d \tilde{U}^a}{ds} + \tilde{Г}_{bc}^a \tilde{U}^b \tilde{U}^c
	\end{align*}

	Можно показать, что данные уравнения содержат в себе как и движение в электромагнитном поле в общей теории относительности, так и силу Лоренца. Не менее интересен ещё один результат, непосредственно вытекающий из данной системы. Выполнение уравнений движения требует того, чтобы удельный заряд частиц $ \frac{q}{m} $ был пропорционален $ \frac{dx^5}{ds_4} $, где в данном случае под $ ds_4 $ подразумевается $ ds_4 = g_{\mu \nu} dx^{\mu} dx^{\nu} $ четырёхмерный интервал. Таким образом, мы получаем, что заряд частицы определяется "скоростью" её движения в пятом измерении.


\subsection{Тензор энергии-импульса}

	В заключение, нужно определить тензор энергии-импульса материи. По аналогии с СТО и ОТО 

	\begin{align*}
	 	\tilde{T}^{ab} = \rho \frac{dx^a}{ds} \frac{dx^b}{ds}
	\end{align*}

	Сравнивая результат такого определения с ОТО, мы понимаем, что пространственные компоненты дают обычным тензор энергии частиц, а компоненты вида $ T^{5 \mu} $ следуют отождествить с энергией электромагнитного поля этих частиц.

\subsection{Cкалярное поле}

	В течение всего обзора мы никак не акцентировали внимание на роли, которую играет в этой теории велицина $ \phi $ составляющая некоторое скалярное поле. Выбирая её константой, мы ограничиваем "электромагнитное поле", так как на связь $ \phi $ и $ F_{\mu \nu} $ существуют дополнительные уравнения. Данный факт не был учтён в ранних работах Калуцы-Клейна, что представляло собой некоторое заблуждение. 

	До сих пор нужное скалярное поле не было обнаружено экспериментально, из-за чего теорию нельзя считать подтверждённой.

\subsection{Развитие}

	На данный момент гипотеза Калуцы-Клейна не считается перспективной, однако, в своё время она послужила началом постороения многомерных теорий объединения и теорий супергравитаций, таких как теория струн или М - теория.


\end{document}