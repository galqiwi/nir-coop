\documentclass{article}
\usepackage[utf8]{inputenc}
\usepackage[russian]{babel}
\usepackage{amsfonts}

\title{Вопрос 4.6}

\begin{document}

\maketitle

\section{Разрешимость}
Уравнение $\left( x^{5}-a^{2}\right) ^{3}=b$ разрешимо в радикалах. В качестве доказательства предоставим его решение:
\\$\left( x^{5}-a^{2}\right) ^{3}=b$
\\$x^{5}-a^{2}=\sqrt [3] {b}e^{\frac {2\pi in}{3}},n\in \left\{ 0,1,2\right\} $
\\$x^{5}=a^{2}+\sqrt [3] {b}e^{\frac {2\pi in}{3}},n\in \left\{ 0,1,2\right\} $
\\$x=\sqrt {a^{2}+\sqrt [3] {b}e^{\frac {2\pi in}{3}}}e^{\frac {2\pi im}{5}},n\in \left\{ 0,1,2\right\} ,m\in \left\{ 0,1,2,3,4\right\} $
\\Таким образом, найдены ровно 15 корней многочлена, который имеет степень, равную 15, то есть все корни выражены через радикалы.

\section{Группа Галуа}
Как известно, уравнение разрешимо тогда и только тогда, когда его группа Галуа разрешима. Покажем это. Расширением Гаула будет являться $\mathbb {L/R}$, где $\mathbb{L} = \mathbb{K}(a ^{2}+\sqrt [3] {b}\left( 1-i\sqrt {3}\right), \: a ^{2}+\sqrt [3] {b}\left( 1+i\sqrt {3}\right)) $, а $\mathbb{K} = \mathbb{R}(i)$.
\\Поле $\mathbb{L}$ разрешимо, поскольку $\mathbb{L} \supseteq \mathbb{K} \supseteq \mathbb{R}$. Фактически, $\mathbb{L/K}$ и $\mathbb{K/Q}$ являются расширениями Галуа. Интересующая нас группа Гаула это $G(\mathbb{L/Q}) = S_4$. Известно также, что $S_n$ при $n < 5$ разрешима, таким образом уравнение действительно разрешимо в радикалах.

\end{document}