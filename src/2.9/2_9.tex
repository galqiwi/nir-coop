\documentclass[a4paper,12pt]{article} % тип документа
% report, book

% Рисунки
\usepackage{graphicx}
\usepackage{wrapfig}

\usepackage{hyperref}
\usepackage[rgb]{xcolor}
\hypersetup{				% Гиперссылки
    colorlinks=true,       	% false: ссылки в рамках
	urlcolor=blue           % на URL
}

%  Русский язык

\usepackage[T2A]{fontenc}			% кодировка
\usepackage[utf8]{inputenc}			% кодировка исходного текста
\usepackage[english,russian]{babel}	% локализация и переносы

% Математика
\usepackage{amsmath,amsfonts,amssymb,amsthm,mathtools} 
\usepackage{wasysym}

\begin{document} % начало документа

2.9
\\Коэффициент диффузии для идеального газа: \[ D=\frac{1}{3}<\lambda><v>=\frac{1}{\sqrt{2} \pi d^2 n} \sqrt{\frac{8 R T}{\pi \mu}},\] где <$\lambda$> - средняя длина свободного пробега молекул, d - диаметр молекул, n - концентрация молекул; <v> - средняя арифметическая скорость молекул, R-газовая постоянная, $\mu$ - молярная масса, T - температура.
\\Коэффициент теплопроводности для идеального газа: \[ \chi=\rho C_{v} D,\] где $C_{v}$ - удельная изохорическая теплоёмкость газа $(C_{v}=\frac{i}{2}\frac{R}{M})$, $\rho$ - плотность газа.
\\Коэффициент вязкости для идеального газа: \[\nu=\rho D,\].

\end{document} % конец документа