\documentclass{article}
\usepackage[utf8]{inputenc}
\usepackage[russian]{babel}
\usepackage{amsmath}
\setlength{\parskip}{1em}
\usepackage{indentfirst}
\usepackage{pgfplots}
\pgfplotsset{compat=1.9}


\begin{document}


\text {\righthyphenmin=2 } {


8. \textbf{Что такое глобальное потепление?}

Глобальное потепление это антропогенное повышение средней температуры климатической системы Земли за длительный промежуток времени. Климатической системой называют объединение атмосферы, гидросферы,криосферы, литосферы и биосферы. Многочисленные независимые наборы данных подтверждают, что десятилетие 2009-2018 годов было на 0,93 $\pm$ 0,07 $^\circ$C теплее доиндустриального периода (1850-1900). В настоящее время температура поверхности повышается примерно на 0,2 $^\circ$C в десятилетие. Климатическая модель прогнозирует, что в 21 веке средняя температура всей поверхности Земли может вырасти еще на 0,3-1,7  $^\circ$C при умеренном сценарии или даже на 2,6-4,8  $^\circ$C при экстремальном сценарии.

Последствия глобального потепления включают повышение уровня моря, региональные изменения осадков, учащение экстремальных погодных явлений, таких как аномальная жара, и расширение пустынь. Закисление океана также вызвано выбросами парниковых газов и обычно группируется с этими последствиями, даже если оно не обусловлено растущей температурой. Наибольшее повышение температуры поверхности наблюдается в Арктике, что способствовало таянию ледников, вечной мерзлоты и морского льда. В целом, повышение температуры приводит к увеличению количества осадков и снегопадов, но в некоторых регионах наоборот засухи и лесные пожары усилятся. Изменение климата угрожает снижением урожайности сельскохозяйственных культур, подрывает продовольственную безопасность, а повышение уровня моря может привести к затоплению прибрежной инфраструктуры и вынудить покинуть многие прибрежные города. Экологические последствия включают исчезновение или перемещение многих видов по мере изменения их экосистем, в первую очередь коралловых рифов, гор и Арктики.

9. \textbf{Чему равна скорость звука в различных средах?}
Скорость звука в жидкости или газе вычисояется по следующей формуле:
$$c = \sqrt{\frac{1}{\beta \rho}}$$

Для воздуха при нормальных условиях
$C_{\text{в}} = 331.46$ м/с

Для воды при нормальных условиях
$C_{\text{h20}} = 1500$ м/с

Скорость звука в кристалле можно посчитать по формуле 
$$v=\sqrt{\frac{k}{m}}d$$
Где d - период кристаллической решетки, k - коэффициент квазиупругой силы, а m масса молекулы (атома).

}

\end{document}
