%FOR PDFLATEX USE ONLY
\documentclass[a4paper,14pt]{article}

\usepackage{amssymb,amsmath} %math symbols

\usepackage[margin=2cm, bottom=2cm]{geometry} %paper geometry

\usepackage[utf8]{inputenc} %allows unicode (including russian) source file
\usepackage[russian]{babel} %docment in russian-style
\usepackage[utf8]{inputenc}
\usepackage[unicode]{hyperref} %links inside of the text
\usepackage[pdftex]{graphicx} %includegraphics pictures
\usepackage{cmlgc} %bold text

\usepackage{array} %arrays

\usepackage{cancel}
\usepackage{wrapfig}
\usepackage{array}
\usepackage{lipsum}
\usepackage{esvect}
\usepackage{longtable}
\usepackage{verbatim} 
\usepackage{multirow}
\usepackage{hyperref}
\usepackage{mathtools}
\usepackage{subfig}
\usepackage{calc}
\usepackage{pgfplots,tikz,circuitikz}
\usepackage{tkz-euclide}
\usepackage{gensymb}



\begin{document}

\section*{2.10}
\begin{center}
	\LARGE{\textbf{Спектр и мощность излучения черного тела в $d + 1$ измерении}}
\end{center}
Излучение абсолютно черного тела в $(n|n > 1)$ пространстве -- известная задача, решение которой человечество уже нашло. Из статьи 2005 года $"$The blackbody radiation in a D-dimensional universes (dx.doi.org/10.1590/S1806-11172005000400007)$"$ видно, что спектр тела в $d+1$ мерном пространстве равен:
\begin{equation*}
	\rho_T(\nu) = 2 \left( \frac{\sqrt{\pi}}{c} \right) ^{d + 1} \frac{d}{\Gamma \left( \frac{d + 1}{2} \right)}\cdot\frac{h\nu^{d + 1}}{exp(h\nu / k_B T) - 1},
\end{equation*}
а энергия излучения на единицу площади равна:
\begin{equation*}
	\sigma_d T^{d + 2},
\end{equation*}
где $\sigma_d$ -- постоянная Стефана — Больцмана $d + 1$ мерного пространства, равная:
\begin{equation*}
	\sigma_d = \left( \frac{2}{c} \right) (\sqrt{\pi})^{d - 1} \frac{k_B^{d+2}}{h^{d + 1}} d (d + 1) \Gamma \left( \frac{D}{2} \right) \zeta(d + 2).
\end{equation*}
\end{document}