\documentclass[12pt]{article}
\usepackage[left=2cm,right=2cm,top=2cm,bottom=2cm,bindingoffset=0cm]{geometry}
\usepackage[T2A]{fontenc}
\usepackage[utf8]{inputenc}
\usepackage[russian]{babel}
\usepackage{amsfonts}
\usepackage{graphicx}
\usepackage{amssymb}
\usepackage{color}
\usepackage{amsmath}
\usepackage{hyperref}
\usepackage{cite}
 
 \title{Задание 1.10.Сколько тепла выделяется при стирании одного бита информации?}
 \begin{document}
 \maketitle
 \textbf{Почему рассеивается энергия?Логическая необратимость и ее связь с дисспациями.}
 
 Назовем устройство логически необратимым ,если по сигналу на выходе нельзя однозначно определить сигнал на входе.Логическая необратимость предполагает физическую необратимость,которая приводит к диссипативным эффектам.
 
 1.Пример
 
 Примером необратимой  логической функции истинности может являться установление в единицу.
 Покажем необратимость. Бинарное устройство представляет собой частицу в бистабильной потенциальной яме.Если бы мы могли перевести частицу из состояния ноль или единица в единицу без потерь энергии,то так как система консервативна,"обратив" время,мы получим систему ,которая по-прежнему будет удволетворять уравнениям движения.В этой системе для одного начального условия результатом будут два состояния(ноль и единица),однако законы механики полностью детерминированы ,и траектория определяется начальным положением и скоростью.
 
 \textbf{Энтропия}
 
 Остается изучить связь между логической необратимостью и \textbf{энтропией}:
 \begin{equation}
     S=k\ln W
 \end{equation}
 ,где k-постоянная Больцмана,W-статистическая вероятность (число микросостояний,реализующее данное макросостояние системы)
 
 При стирании число состояний уменьшается в 2 раза ,то есть энтропия уменьшается на $kln2$.
 Энтропия замкнутой системы не может уменьшаться,следовательно она должна проявиться в виде эффекта нагревания,тогда выделится $E\succeq k T \ln 2$ \textbf{(Принцип Ландауэра).}
 
 Ограничения ,накладываемые принципом Ландауэра можно обойти путём реализации обратимых вычислений, при этом возрастают требования к объёму памяти и количеству вычислений.
 
 \end{document}