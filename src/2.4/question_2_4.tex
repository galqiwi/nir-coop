\documentclass[a4paper, 12pt]{article}
\usepackage[T2A,T1]{fontenc}
\usepackage[utf8]{inputenc}
\usepackage[english, russian]{babel}
\usepackage{graphicx}
\usepackage[hcentering, bindingoffset = 10mm, right = 15 mm, left = 15 mm, top=20mm, bottom = 20 mm]{geometry}
\usepackage{multirow}
\usepackage{lipsum}
\usepackage{amsmath, amstext}
\usepackage{siunitx}
\usepackage{mathrsfs}
\usepackage{subcaption}
\usepackage{wrapfig}
\usepackage{adjustbox}
\usepackage{enumerate, indentfirst, float}
\usepackage{capt-of, svg}
\usepackage{icomma}
\newenvironment{bottompar}{\par\vspace*{\fill}}{\clearpage}
\begin{document}
\title{Вопрос 2.4, Хронология событий во Вселенной}

\begin{enumerate}
\item \textbf{Планковская эпоха}  $0 - 10^{-43}$ c

Момент, с которого началась физика. После Планковской эпохи гравитационное взаимодействие отделилось от отстальных.
\item \textbf{Эпоха великого объединения} $10^{-43} - 10^{-34}$ c

С момента начала ЭВО квантовые эффекты слабеют и вступают в силу законы ОТО.

\item \textbf{Эпоха раздувания} $10^{-36} - 10^{-32}$ c

В эту эпоху Вселенная всё ещё преимущественно заполнена излучением, начинают образовываться кварки, электроны и нейтрино.

\item \textbf{Эпоха электрослабых взаимодействий} $10^{-32} - 10^{-12}$ c

За счёт очень высоких энергий образуется ряд экзотических частиц, таких как бозон Хиггса и W-бозон, Z-бозон.

\item \textbf{Эпоха кварков} $10^{-12} - 10^{-6}$ c

Электромагнитное, гравитационное, сильное, слабое взаимодействия формируются в их современном состоянии.

\item \textbf{Эпоха адронов} $10^{-6} - 100$ c

Кварк-глюонная плазма охлаждается, и кварки начинают группироваться в адроны, включая, например, протоны и нейтроны.

\item Эпоха лептонов 100 c - 3 мин

В ходе адронной эпохи большая часть адронов и антиадронов аннигилируют (взаимоуничножаются) друг с другом и оставляют пары лептонов и антилептонов преобладающей массой во Вселенной. Лептоны и антилептоны, в свою очередь аннигилируют друг с другом и во Вселенной остаётся лишь небольшой остаток лептонов.

\item \textbf{Протонная эпоха} 3 мин - 380 000 лет

материя охладилась достаточно для образования стабильных нуклонов и начался процесс первичного нуклеосинтеза. за это время образовался первичный состав звёздного вещества: около 25 $\%$ гелия-4, 1 $\%$ дейтерия, следы более тяжёлых элементов до бора, остальное — водород.

\item \textbf{Темные века} 380 000 лет - 550 млн лет

Вещество начинает доминировать над излучением (70 000 лет), что приводит к изменению режима расширения Вселенной. В конце эпохи (380 000 лет) происходит рекомбинация водорода и Вселенная становится прозрачной для фотонов теплового излучения.

\item \textbf{Реонизация} 550 млн лет - 800 млн лет

Образуются первые звёзды, галактики, квазары, скопления и сверхскопления галактик. Реионизация водорода светом звёзд и квазаров.

\item \textbf{Эра вещества} 800 млн лет - сейчас

Образование межзвёздного облака, давшего начало Солнечной системе. Образование Земли и других планет нашей Солнечной системы, затвердевание пород.
\end{enumerate}
\end{document}