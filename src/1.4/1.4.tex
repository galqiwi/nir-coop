\documentclass{article}
\usepackage{amsmath}
\usepackage{mathtext}
\usepackage[T1,T2a]{fontenc}
\usepackage[utf8]{inputenc}
\usepackage[english, russian]{babel}
\usepackage{tikz}
\usepackage{longtable}
\usepackage{verbatim} 
\usepackage{multirow}
\usepackage{pgfplots}
\usepackage[export]{adjustbox}
\usepackage[left=2cm,right=2cm,
    top=2cm,bottom=2cm,bindingoffset=0cm]{geometry}

% объявляем новую команду для переноса строки внутри ячейки таблицы
\newcommand{\specialcell}[2][c]{%
  \begin{tabular}[#1]{@{}c@{}}#2\end{tabular}}
\newenvironment{comment}{}{}

\begin{document}
\section{Последовательность событий в космологии}
Возраст вселенной по разным данным $13.75 \pm 0.13$ (WMAP) или $13.81 \pm 0.06$ (Planck) гигалет.
\begin{center}
\begin{longtable}{|c|c|c|}
\hline
 Время& Эпоха& События\\
\hline
 0& Сингулярность& Большой взрыв\\
\hline
 0 — $10^{-43}$ с& Планковская эпоха& Рождение частиц\\
\hline
 $10^{-43}$ — $10^{-35}$ с & Эпоха Великого объединения& \specialcell{Отделение гравитации от\\ объединённого электрослабого\\ и сильного взаимодействия.\\ Возможное рождение\\ монополей. Разрушение\\ Великого объединения.} \\
\hline
 $10^{-35}$ — $10^{-32}$ с& Инфляционная эпоха& \specialcell{Вселенная экспоненциально\\ увеличивает свой радиус на\\ много порядков. Структура\\ первичной квантовой\\ флуктуации, раздуваясь, даёт\\ начало крупномасштабной\\ структуре Вселенной.\\ Вторичный нагрев.} \\
\hline
 $10^{-32}$ — $10^{-12}$ с& Электрослабая эпоха& \specialcell{Вселенная заполнена кварк-\\глюонной плазмой, лептонами,\\ фотонами, W- и Z-бозонами,\\ бозонами Хиггса. Нарушение\\ суперсимметрии.} \\
\hline
 $10^{-12}$ — $10^{-6}$ с& Кварковая эпоха& \specialcell{Электрослабая симметрия\\ нарушена, все четыре\\ фундаментальных\\ взаимодействия существуют\\ раздельно. Кварки ещё не\\ объединены в адроны.\\ Вселенная заполнена кварк-\\глюонной плазмой, лептонами и\\ фотонами.} \\
\hline
 $10^{-6}$ — 100 с& Адронная эпоха& \specialcell{Адронизация. Аннигиляция\\ барион-антибарионных пар.\\ Благодаря CP-нарушению\\ остаётся малый избыток\\ барионов над антибарионами\\ (около 1:$10^9$).} \\
\hline
 100 секунд — 3 минуты& Лептонная эпоха& \specialcell{Аннигиляция лептон-\\антилептонных пар. Распад\\ части нейтронов. Вещество\\ становится прозрачным для\\ нейтрино.} \\
\hline
 3 минуты — 380 000 лет& Протонная эпоха& \specialcell{Нуклеосинтез гелия, дейтерия,\\ следов лития-7 (20 минут).\\ Вещество начинает\\ доминировать над излучением\\ (70 000 лет), что приводит к\\ изменению режима расширения\\ Вселенной. В конце эпохи (380\\ 000 лет) происходит\\рекомбинация водорода и\\ Вселенная становится\\ прозрачной для фотонов\\ теплового излучения.} \\
\hline
 380 000 — 550 млн лет& Тёмные века& \specialcell{Вселенная заполнена\\ водородом и гелием,\\ реликтовым излучением,\\ излучением атомарного\\ водорода на волне 21 см.\\ Звёзды, квазары и другие яркие \\источники отсутствуют.} \\
\hline
 550 млн — 800 млн лет& Реионизация& \specialcell{Образуются первые звёзды\\ (звёзды популяции III), квазары,\\ галактики, скопления и\\ сверхскопления галактик.\\ Реионизация водорода светом\\ звёзд и квазаров.} \\
\hline
 800 млн лет— 8,9 млрд лет& Эра вещества& \specialcell{Образование межзвёздного\\ облака, давшего начало\\ Солнечной системе.} \\
\hline
 8,9 млрд лет — 9,1 млрд лет& Эра вещества& \specialcell{Образование Земли и других\\ планет нашей Солнечной\\ системы, затвердевание пород.} \\
\hline
\end{longtable}
\end{center}
\end{document}