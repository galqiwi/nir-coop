\documentclass[12pt]{article}
\usepackage[left=2cm,right=2cm,top=2cm,bottom=2cm,bindingoffset=0cm]{geometry}
\usepackage[T2A]{fontenc}
\usepackage[utf8]{inputenc}
\usepackage[russian]{babel}
\usepackage{amsfonts}
\usepackage{graphicx}
\usepackage{amssymb}
\usepackage{color}
\usepackage{amsmath}
\usepackage{hyperref}
\usepackage{cite}
 
 \title{Задание 5.3 Почему уравнения движения для фермионов первого порядка?}
 \begin{document}
 \maketitle
 Движение фермионов описывается уравнениями Дирака,которые являются релятивистским обобщением \textbf{уравнения Шредингера}:
 \begin{equation}
 i \hbar \frac{\partial}{\partial t} \Psi=\hat{H}(p, q) \Psi
 \end{equation}
 В уравнении Шредингера стоит первая производная по времени, потому что по теореме Нетер гамиальтониан-величина,сохраняющаяся при инвариантости системы относительно трансляций по времени(а генератором трансляций по времени выступает оператор производной по времени)
 
 Однако уравнение Шредингера не удовлетворяет принципам СТО так как оно не инвариантно относительно преобразований Лоренца(тк вторые производные по координатам и первая по времени).Поэтому его релятивистское обобщение имеет вид:
 \begin{equation}
 i \hbar \frac{\partial \psi}{\partial t}=\left[c\left(\alpha_{x} \frac{\partial}{\partial x}+\alpha_{y} \frac{\partial}{\partial y}+\alpha_{z} \frac{\partial}{\partial z}\right)+\beta m c^{2}\right] \psi
 \end{equation}
 \end{document}